\chapter{Conclusions \& Future Work}
\label{ch: summary}
 %General
The formation of stars is one of the fundamental factors in the formation and evolution of galaxies.
Measuring the star formation rate and its variation through time helps us to learn about how a galaxy forms and evolves.
Many efforts have been made to find a universal law for the rate at which stars form.
In order to find a universal law, finding a method to measure SFR accurately is the first step.
Then correlations between SFR and other properties of galaxies must be investigated.
Afterward, each law has to be tested in various environments to confirm its universality.
Therefore, we need high quality data and good statistical methods to test the laws.

It has been well established that to form stars, gas is the main ingredient.
However, the relation between SFR and different tracers of gas mass (atomic, molecular, or total gas) changes in different type of galaxies.
Relations between stellar mass, metallicity and SFR are not clear and there are fewer studies of these than of the relation between SFR and gas mass.
The dependence of SFR on other properties of galaxies such as morphology and dust mass have been tested as well, but there is no SFR law that considers all of the above quantities together.
The Kennicutt-Schmidt law ~\citep{Schmidt59, Kennicutt98b} was the first empirical star formation law, suggesting that SFR and molecular gas have a logarithmic correlation with each other. 
However, after many years and studies, there is still an ongoing question of whether it is a universal law or not.
\citet{Shetty13} showed that, besides the physical and environmental effects, fitting methods also change our perspective of the K-S law.
Extended version of the K-S law were introduced by adding the effect of stars (extended Schmidt law; introduced by ~\citealt{Shi11}) or metallicity (Krumholz law; introduced by ~\citealt{Krumholz09}).
Compared to the K-S law, the extended Schmidt law and Krumholz law are relatively recently introduced and still need to be tested with diverse statistical methods on different scales (local and global), with various morphological types and in both nearby and high-redshift galaxies.

%Chapter 2
Spatially resolved images of the Andromeda galaxy (M31) cover different types of environments within a galaxy: high or low stellar surface density, gas mass, and metallicity.
These images permit us to examine star formation laws in detail, in various environments.
In Chapter~\ref{ch: paper1}, we investigated three star formation laws in M31 on both local and global scales in order to determine which of these laws describe M31 data more accurately.
To achieve this goal, we produced maps of surface density of SFR, gas mass, stellar mass and metallicity of the galaxy.
We created SFR maps using three SFR indicators: a combination of the FUV and 24~\um emission, a combination of the \halpha and 24~\um emission, and TIR luminosity. 
Using a combination of the FUV and 24~\um emission, we determined the total SFR for M31 to be $0.31\pm 0.04$~M$_\odot$~yr$^{−1}$.
We measured the ISM gas mass using molecular gas only, atomic gas only and the total gas.
We confirmed that the extended Schmidt law holds in M31 for local regions and the effect of stellar mass in regions with low gas density is even more pronounced. 
By applying both the K-S law and the extended Schmidt law on all SFR maps and gas maps, we concluded that changes in SFR tracer do not affect these laws as much as changes in gas tracer.
Our fitting results show that in regions with higher star formation rate, power laws are more in agreement with other studies.
In order to examine Shetty's suggestions on the effect of fitting methods on SFR laws, we used hierarchical Bayesian regression fitting and normal regression fitting to fit SFR laws.
The results from the two different methods are dissimilar, mostly because of the ways these two methods handle the uncertainties.
The measured SFR by the Krumholz law did not match with the observed data in M31.
By performing statistical tests, we showed that there is no strong correlation between metallicity and SFR in M31.


%Chapter 3 
For the remainder of this thesis we used a machine learning method to have a fresh look on star formation and its relation to other properties of galaxies.
We chose the self-organizing map algorithm as our data mining method because of its ability to classify results and show the morphology of data, simultaneously.
The amount of observational data and derived quantities for M31 is enough to make this galaxy a good target for machine learning studies.
Therefore, in Chapter~\ref{ch: paper3} we applied the self-organizing map algorithm on observational data from M31.
We used observational data and derived quantities of 10 regions in M31, which were chosen due to the availability of mid-infrared spectroscopy for those regions.
Therefore, we were able to focus on relations between SFR and interstellar medium components, specifically polycyclic aromatic hydrocarbons (PAHs).
Using a network with 2 neurons, we divided the data into two major groups.
By comparing data from these groups, we found correlations within groups that otherwise would not have been seen.
In one of the groups, the flux from PAHs showed strong anti-correlations with \halpha, \sii, \oiii~and IRAC 5.8~$\mu$m emission, stellar mass and radiation hardness index.
We found two reasons for PAHs to anti-correlate with optical emission and RHI.
First, the harder radiation field could make the size of \hii~regions bigger and the size of photo-dissociation regions smaller, causing more \halpha~emission and less PAH emission.
Second, since optical emission was not corrected for dust extinction, dustier regions would have less observed optical emission.
Since for 10 regions in M31 we needed at least 14 neurons to separate all 10 regions from each other, we concluded that some of these 10 regions have very similar properties.
Using subsets of the data we generated various two dimensional networks.
The networks created by the subsets showed that, except for PAH-only subsets, region 10, which is located in the bulge of M31, becomes isolated.
The network from the PAH-only data is the only network with no isolated regions, which means that, despite differences in underlying physical conditions in different regions, PAHs in M31 are very similar.
We created another network using the same set of M31 observations that were available for M101.
Using subsets of data, we show that the SOMs can be used to predict observations for regions that lack the observational data for some quantities (e.g.\ PAH emission).
We applied this network to the M101 observations and showed that regions with relative similarity to M31 data were placed in the same neurons.
Therefore, we can use these networks to predict properties of other galaxies.


%chapter 4
In Chapter~\ref{ch: paper2}, we used the self-organizing map algorithm on data from high-redshift galaxies.
We classified the template spectra of~\citet{Kinney96}, made from galaxies with known morphological type, and created networks with different uses.
Similar to the results of Chapter~\ref{ch: paper3}, since we needed at least 22 neurons to separate 12 spectral types, we concluded that some of the spectral types have very similar features (types B and E, and types SB1 and SB2 from \citealt{Kinney96}).
We showed that network generated by the self-organizing map method can be used to identify new types of spectra in large surveys.
Using the trained networks, we classified a sample of 142 galaxies from~\citet{Hossein12}.
The classification by self-organizing map allowed us to identify galaxies with SEDs similar to two or more morphological types.
Comparing our results with classifications from trained networks showed that this unsupervised method could classify all 142 spectra while the supervised methods failed to classify 37 out 142 galaxies. 
The properties of the 142 classified galaxies show better correlations in mean values of age, specific star formation rate, stellar mass, and far-UV extinction compared to the other methods.
These correlations are also in agreement with the galaxies' morphological types.

%future works and add big questions
In this thesis we first measured star formation rate in a nearby galaxy and tested the the current star formation laws using Bayesian statistic on the galaxy.
We used data mining methods study  observed or measured quantities of the nearby galaxies and compare them with the SFR.
We then applied the spectra of high-redshift galaxies to determine the correlation between SFR with properties of galaxies in a distant galaxies and through the time.
To move forward, we would like to test star formation laws in high-redshift galaxies with the hierarchical Bayesian regression fitting method. This would allow investigation of both the effects of fitting methods on using global values for galaxies and differences in using this method on nearby galaxies and high-redshift ones.  
We could also use data mining methods to recalibrate the correlations between luminosities from star forming regions and the SFR in both nearby and high redshift galaxies.
Self-organizing maps can help us to examine our calibrations of the SFR, find their flaws and predict the SFR in distant galaxies.
Another project to continue after this thesis would be to show that self-organizing maps are good tools to predict properties of galaxies and estimate values of unobserved quantities.
Continuing the work in Chapter~\ref{ch: paper3}, we would like to obtain mid-infrared spectroscopy with more coverage in M31 and other nearby galaxies to have a higher dimensional sample for creating SOMs as well as to investigate the anti-correlation between stellar mass and PAHs.
Having these new calibrations for finding properties of galaxies and applying them to observations from recently (or soon to be) operational telescopes such as TMT, ALMA, SKA, and {\it JWST} would help us solve the mystery of how stars form and evolve in time.




\addcontentsline{toc}{section}{Bibliography}
\bibliographystyle{apj.bst}
\bibliography{western_bib.bib}